\section{Appendix: Algorithm Options Script (Interim)}

\begin{mdframed}[linecolor=black, topline=true, bottomline=true,
  leftline=false, rightline=false]
\begin{minted}[linenos,breaklines]{csharp}
using UnityEngine;
using UnityEngine.UI;
using System.Collections;
using System;

public class VortonOptionsScript : MonoBehaviour {

    //Declaring Input Variables
    public InputField rows;
    public InputField columns;
    public Button spawnButton;
    public Button clearButton;
    public GameObject vortexProbeParticle;
    public Transform probePointCollection;

    //Declaring spawn related variables
    public float vortonSpacing;
    public float vortonStrength;
    private Vector3 vortonPosition;
    public Vector3 vortonVelocity;

    //Conversion related vaiables
    public int rowsInt;
    public int columnsInt;

    //Array to hold blobs
    private GameObject[] blobArray;

    // Use this for initialization
    void Start () {

        //Add click events
        spawnButton.onClick.AddListener(SpawnButton);
        clearButton.onClick.AddListener(ClearButton);

    }
	
	// Update is called once per frame
	void Update () {

    }

    void SpawnButton() {

        //Button even indicator


        //Convert InputField strings to Integers
        rowsInt = Convert.ToInt32(rows.text);
        columnsInt = Convert.ToInt32(columns.text);

        //Two for loops to produce vortons in a grid shape
        for (int x = 0; x < rowsInt; x++)
        {
            for (int y = 0; y < columnsInt; y++)
            {

                //Produce necessary variables for the instantiated vortons
                vortonPosition = new Vector3(x * vortonSpacing, 1, y * vortonSpacing);

                //Produce actual Vortons
                GameObject clone = Instantiate(vortexProbeParticle,vortonPosition, Quaternion.identity) as GameObject;

                //Puts the new blobs in a eay to handle place
                clone.transform.parent = probePointCollection;

                //Sets some random values for vortex parameters so the result math isnt zero
                VortexBlobBehaviour scr = clone.GetComponent<VortexBlobBehaviour>();  //Get the blob behaivour script
                scr.VortexBlobStrength = vortonStrength;
                scr.initialisationVelocity = vortonVelocity;
                scr.initialVelocity = Vector3.zero;

            }

        }

    }

    void ClearButton()
    {

        //Populate array with current blobs
        blobArray = GameObject.FindGameObjectsWithTag("VortexBlob");

        //Delete all current blobs
        for (int id = 0; id <blobArray.Length; id++)
        {
            Destroy(blobArray[id]);
        }

    }

}
\end{minted}
\end{mdframed}